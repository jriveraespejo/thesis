\chapter{Application} \label{chap:application}

\section{Objectives}

Supported by the theoretical framework of measurement and analysis, the current application has the purpose to apply the proposed model, in order to assess:
\begin{enumerate}
	\item Pychometric properties of the items and texts, texts is an extra
	\item related to the hypothesis of regression variables
	\item Retrodictive accuracy, and concurrent analysis: how the classified poeple is characterized, math scales vs the scores (cut-of of 30 points, or 15 items, 2 points / item)
\end{enumerate}

to a real data set corresponding to 


La presente aplicación se apoya en este marco teórico de medición y análisis, y tiene
como principal propósito el aplicar el modelo investigado a un conjunto de datos reales; y
como propósitos secundarios comprobar si el modelo permite hallar un ordenamiento en las
categorı́as inicialmente no ordenadas, si brinda información relevante para valorar la calidad
de un ı́tem y sus alternativas, y finalmente, corroborar si es útil para caracterizar a los
individuos expuestos al instrumento de evaluación; es decir, encontrar perfiles de docentes.
El modelo fue aplicado en un conjunto de datos que se caracterizó por:
1. Una muestra anónima, aleatoria y representativa de 1641 docentes del nivel secundario
de la especialidad de Inglés, modalidad de Educación Básica Regular (EBR), evalua-
dos en el primer concurso de Ingreso a la Carrera Publica Magisterial y Contratación
Docente aplicado en el 2015 (en adelante Nombramiento 2015); y
2. Un conjunto de 25 ı́tems correspondientes a la sub-prueba de Comprensión de Textos
del mismo concurso.

%%%%%%%%%%%%%%%%%%%%%%%%%%%%%%%%%%%%%%%%%%%%%%%%%%%%%%%%%%%%%%%%%%%%%%%

\section{Instruments}


%%%%%%%%%%%%%%%%%%%%%%%%%%%%%%%%%%%%%%%%%%%%%%%%%%%%%%%%%%%%%%%%%%%%%%%
%%%%%%%%%%%%%%%%%%%%%%%%%%%%%%%%%%%%%%%%%%%%%%%%%%%%%%%%%%%%%%%%%%%%%%%

\section{Data}

\subsection{Collection}

%%%%%%%%%%%%%%%%%%%%%%%%%%%%%%%%%%%%%%%%%%%%%%%%%%%%%%%%%%%%%%%%%%%%%%%

\subsection{Sample scheme}


%%%%%%%%%%%%%%%%%%%%%%%%%%%%%%%%%%%%%%%%%%%%%%%%%%%%%%%%%%%%%%%%%%%%%%%

\section{Results}

All of the preceding suggests one way to navigate overfitting and underfitting: Evaluate our models out-of-sample. using cross validation and information criteria

Pareto-smoothed importance sampling cross-validation (PSIS) and the  Widely Applicable Information Criterion (WAIC).