\chapter{Application} \label{chap:application}

\section{Objectives}

Considering the GLLAMM model developed in previous chapters, the current section has a four-fold purpose, related to its application on a real data set:

\begin{enumerate}
	%
	\item \textbf{Performance of the parametrizations.} In line with the previous chapter, the researcher wants to assess if changing the posterior sampling geometries would benefit the performance of the MCMC method.
	%
	\item \textbf{Assess psychometric properties.} As in any standardized evaluation, instrument developers has a special interest in determine how difficult were the items/texts, with the purpose 
	%
	\item \textbf{Evaluate hypothesis.} related to the hypothesis of regression variables
	%
	\item \textbf{Retrodictive accuracy.} Retrodictive accuracy, and concurrent analysis: how the classified poeple is characterized, math scales vs the scores (cut-of of 30 points, or 15 items, 2 points / item)
	%
\end{enumerate}

to a real data set corresponding to 

corroborar si es útil para caracterizar a los
individuos expuestos al instrumento de evaluación; es decir, encontrar perfiles de docentes.


%%%%%%%%%%%%%%%%%%%%%%%%%%%%%%%%%%%%%%%%%%%%%%%%%%%%%%%%%%%%%%%%%%%%%%%

\section{Instruments}

El instrumento completo consistió en una evaluación estandarizada de 90 preguntas de
opción múltiple, con cuatro alternativas por ı́tem, organizados en tres sub-prueba: (i) Com-
prensión Lectora, a la que le corresponden los primeros 25 ı́tems, (ii) Razonamiento
Lógico, evaluados en los siguientes 25 ı́tems y (iii) Conocimientos Pedagógicos y de la
Especialidad, evaluado en los últimos 40 ı́tems.
La sección elegida para ser analizada bajo el modelo estudiado fue la de Comprensión
Lectora. La sub-prueba fue diseñada para evaluar la capacidad que un docente posee para
reconstruir el significado de diversos tipos de textos en distintos formatos. La sub-prueba
evaluó tres dominios:

Comprensión literal : Capacidad para ubicar información explı́cita en textos comple-
jos,
Comprensión inferencial : Capacidad para integrar información de un texto con el
objetivo de inferir su temática, propósito o relaciones lógicas implı́citas en sus distintos
componentes,
Reflexión sobre el texto: Capacidad de reflexionar crı́ticamente sobre el contenido
y forma de textos diversos.

%%%%%%%%%%%%%%%%%%%%%%%%%%%%%%%%%%%%%%%%%%%%%%%%%%%%%%%%%%%%%%%%%%%%%%%
%%%%%%%%%%%%%%%%%%%%%%%%%%%%%%%%%%%%%%%%%%%%%%%%%%%%%%%%%%%%%%%%%%%%%%%

\section{Data}

\subsection{Collection}

Para tener acceso a los datos referidos se siguió el trámite de acceso a la información en
la sede principal del Ministerio de Educación (MINEDU).

%%%%%%%%%%%%%%%%%%%%%%%%%%%%%%%%%%%%%%%%%%%%%%%%%%%%%%%%%%%%%%%%%%%%%%%

\subsection{Sample scheme}

Con el fin de comparar los estimados que se obtengan del modelo NRM, proveniente de
los métodos MML y MCMC, se accedió a una muestra anónima, aleatoria y representativa
de 1641 docentes que fueron expuestos a la evaluación de interés.

El modelo fue aplicado en un conjunto de datos que se caracterizó por:
1. Una muestra anónima, aleatoria y representativa de 1641 docentes del nivel secundario
de la especialidad de Inglés, modalidad de Educación Básica Regular (EBR), evalua-
dos en el primer concurso de Ingreso a la Carrera Publica Magisterial y Contratación
Docente aplicado en el 2015 (en adelante Nombramiento 2015); y
2. Un conjunto de 25 ı́tems correspondientes a la sub-prueba de Comprensión de Textos
del mismo concurso.


%%%%%%%%%%%%%%%%%%%%%%%%%%%%%%%%%%%%%%%%%%%%%%%%%%%%%%%%%%%%%%%%%%%%%%%

\section{Results}

\subsection{Parametrization performance}

of were wants to assess  Pychometric properties of the items and texts, texts is an extra

All of the preceding suggests one way to navigate overfitting and underfitting: Evaluate our models out-of-sample. using cross validation and information criteria

Pareto-smoothed importance sampling cross-validation (PSIS) and the  Widely Applicable Information Criterion (WAIC).