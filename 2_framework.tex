\chapter{The Generalized Linear Latent and Mixed Model} \label{chap:framework}

The Generalized Linear Latent and Mixed Model (GLLAMM) is a framework that unifies a wide range of latent variable models. Developed by \citet{Rabe_et_al_2004a, Rabe_et_al_2004b, Skrondal_et_al_2004}, the method was motivated by the need of multilevel Structural Equation Models (multilevel SEM) to accommodate for unbalanced data, noncontinuous responses and the use of cross-level effects among latent variables. \\

This chapter will present the definition of such model, its characteristics, assumptions and properties.


\section{Definition}

In the context of a standardize assessment, we assume that $J$ subjects answer $I$ items, where the response to each one of them is assumed to be explained by one or $M_{l}$ latent variables at $L$ arbitrary levels. Following \citet{Rabe_et_al_2004a}, we depart from the traditional multivariate framework for formulating factor and structural models, i.e. a wide format, and adopt a univariate approach, i.e. long format. In that sense, all the response variables for each subject are "stacked" in a single response vector with different variables distinguished from each other by a design matrix. \\

With the aforementioned structure, we proceed to outline the three parts of the framework: (i) the response model, (ii) the structural latent variable model, and (iii) the distribution of the latent variables. For a detailed description of the special cases of multilevel SEM that can be derived with the framework, refer to the Appendix \ref{app:additional}.


\subsection{Response model}

As outlined in \citet{Rabe_et_al_2004a, Rabe_et_al_2012}, conditional on the latent variables, the response model is a generalized linear model (GLM) defined by a linear predictor, a link function and a distribution from the exponential family.

\subsubsection{Linear predictor}

For a model with $L$ arbitrary levels and $M_{l}$ latent variables at $l>1$ levels, the linear predictor for subject $j$ takes the following form:

\begin{equation}
	\mathbf{v}_{j} = \mathbf{X}_{j} \pmb{\beta} + \sum_{l=2}^{L} \sum_{m=1}^{M_{l}} \eta_{mj}^{l} \mathbf{Z}_{mj}^{l} \pmb{\lambda}_{m}^{l}
\end{equation}

\noindent where $\mathbf{v}_{j}$ is the vector of linear predictor for subject $j$ ($j=1, \dots, J$), $\mathbf{X}_{j}$ is a design matrix that maps the parameter vector $\pmb{\beta}$ to the linear predictor, $\eta_{mj}^{l}$ is the $m$th latent variable at level $l$ ($m=1, \dots, M_{l}$ and $l=1, \dots, L$), and $\mathbf{Z}_{mj}^{l}$ is a design matrix that maps the vector of loadings $\pmb{\lambda}_{m}^{l}$ to the $m$th latent variable at level $l$.

\subsubsection{Links and distributions}

\begin{equation}
	g(E[\mathbf{Y} | \mathbf{X}, \mathbf{Z}, \pmb{\eta}]) = \mathbf{V}
\end{equation}

\noindent where the distributions assumed for the responses depends on their type:
\begin{enumerate}
	\item \textbf{Continuous:}
	\item \textbf{Ordinal and discrete time duration:}
	\item \textbf{Dichotomous:}
	\item \textbf{Counts and continuous time duration:}
	\item \textbf{Polytomous:}
	\item \textbf{Mixed responses:}
\end{enumerate}

\subsection{Structural model for the latent variables}

\subsection{Distribution of the latent variables}



\section{Relationship with other modeling schemes}

\citet{Rabe_et_al_2012}

{\color{red} The motivation for multilevel regression models is to handle hierarchical data where elementary units are nested in clusters, such as students in schools, which in
turn may be nested in higher-level clusters (e.g., school districts or states). The latent variables, often called "random effects” in this context, can be interpreted as
the effects of unobserved covariates at different levels that induce dependence among lower-level units. In contrast, the motivation for structural equation models is to
handle variables that cannot be measured directly, and are hence latent, and to model their relationships with each other and with observed or manifest variables.
The latent variables, often called “common factors” inthis context, are measured by manifest variables and induce dependence among them.}


\subsection{Factor Models}

\subsection{Item Response Theory and Generalized Latent Models}

\subsection{Multilevel Models}



\section{Advantages and Disadvantages}