\chapter{The Generalized Linear Latent and Mixed Model} \label{chap:framework}

The Generalized Linear Latent and Mixed Model (GLLAMM) is a framework that unifies a wide range of latent variable models. Developed by \citet{Rabe_et_al_2004a, Rabe_et_al_2004c Skrondal_et_al_2004}, the method was motivated by the need of a multilevel Structural Equation Models (SEM) that accommodates for unbalanced data, noncontinuous responses and the use of cross-level effects among latent variables. 

This chapter will present the definition of such model, its characteristics, assumptions and properties.


\section{Definition}
Following \citet{Rabe_et_al_2004a, Rabe_et_al_2012}, we depart from the traditional multivariate framework for formulating factor and structural models, i.e. a "wide" format, and adopt a univariate approach, i.e. "long" format. In that sense, all the response variables for each unit are "stacked" in a single response vector with different variables distinguished from each other by a design matrix.

With the aforementioned structure, we proceed to outline the three parts of the framework: (i) the response model, (ii) the structural latent variable model, and (iii) the distribution of the latent variables. For a detailed description of some of the special cases of multilevel SEM that can be derived with this framework, refer to Appendix \ref{appA:additional}.


\subsection{Response model}
As outlined by the authors, conditional on the latent variables, the response model is a generalized linear model (GLM) defined by a systematic and a distributional part. For the systematic part, a linear predictor and a link function are selected. Finally, for the distributional part, a distribution from the exponential family is also selected. 

In the following sections, we proceed to describe the linear predictor, the link function and the distributions accommodated by the framework.


\subsubsection{Linear predictor}
For a model with $L$ levels and $M_{l}$ latent variables at $l>1$ levels, the linear predictor takes the following form:
\begin{equation} \label{eq:linear_predictor}
	v = \mathbf{X} \pmb{\beta} + \sum_{l=2}^{L} \sum_{m=1}^{M_{(l)}} \eta_{m}^{(l)} \mathbf{Z}_{m}^{(l)} \pmb{\lambda}_{m}^{(l)}
\end{equation}

\noindent where $\mathbf{X}$ is a design matrix that maps the parameter vector $\pmb{\beta}$ to the linear predictor, $\eta_{m}^{(l)}$ the $m$th latent variable at level $l$ ($m=1, \dots, M_{(l)}$ and $l=1, \dots, L$), and $\mathbf{Z}_{m}^{(l)}$ a design matrix that maps the vector of loadings $\pmb{\lambda}_{m}^{(l)}$ to the $m$th latent variable at level $l$.

Note that wo do not use subscripts for the units of observation at different levels. This decision was made with the purpose of avoiding the use of mathematical definitions with large number of subscripts. However, a careful reader should consider that equation \ref{eq:linear_predictor} rest on the assumption that each unit is identified at their appropriate level. For special cases of multilevel SEM and their use of subscripts refer to Appendix \ref{appA:additional}.

\subsubsection{Links and Distributions}
The model "links" the expectation of the conditional response to the linear predictor through a response function $h(\cdot)$, in the following form: 
\begin{equation} \label{eq:response_function}
	\mu = E[y | \mathbf{X}, \mathbf{Z}, \pmb{\eta}] = h(v)
\end{equation}

\noindent where equation \ref{eq:response_function} can be re-written in terms of the link function $g(\cdot) = h^{-1}(\cdot)$:
\begin{equation} \label{eq:link_function}
	g(\mu) = g(E[y | \mathbf{X}, \mathbf{Z}, \pmb{\eta}]) = v
\end{equation}

\noindent with $\pmb{\eta}=\left[\eta^{(2)T}, \dots, \eta^{(L)T}\right]^{T}$ and $\mathbf{Z}=\left[\mathbf{Z}^{(2)T}, \dots, \mathbf{Z}^{(L)T}\right]^{T}$, as the "stacked" vector of latent variables, and the "stacked" design matrices of explanatory variables, for all $L$ levels, respectively. Additionally, $\pmb{\eta}^{(l)}=\left[\eta_{1}^{(l)}, \dots, \eta_{M_{(l)}}^{(l)}\right]^{T}$ and $\mathbf{Z}^{(l)}=\left[\mathbf{Z}_{1}^{(l)T}, \dots, \mathbf{Z}_{M_{(l)}}^{(l)T}\right]^{T}$, denotes the vector of latent variables, and the "stacked" design matrix of explanatory variables, at level $l$, respectively.

Finally, the response model specification is complete when we select an appropriate distribution from the family of exponential distributions. The types of responses that can be accommodated by the framework are the following:

\begin{enumerate}
	
	\item \textbf{Continuous:} \\
	It results form selecting an identity link function for the mean response,
	\begin{equation} \label{eq:link_cont}
	\mu = v
	\end{equation}
	On the other hand, the distributional part use an standard normal distribution,
	\begin{equation} \label{eq:dist_cont}
		\begin{split}
		f(y| \mathbf{X}, \mathbf{Z}, \pmb{\eta}) &= \phi(\mu) \\
		&= \phi(v)
		\end{split}
	\end{equation}
	where $\phi(x)$ describes the Standard Normal density distribution. 
	
	
	
	\item \textbf{Dichotomous:} \\
	It results from selecting an appropriate response function for the expected value of the manifest variable, which describe the probability of endorsing one of the two available categories,
	\begin{equation} \label{eq:link_dich}
		\begin{split}
		\mu &= E[y=1 | \mathbf{X}, \mathbf{Z}, \pmb{\eta}] \\ 
		&= P[y=1 | \mathbf{X}, \mathbf{Z}, \pmb{\eta}] \\
		&= \pi \\
		&= h(v)
		\end{split}	
	\end{equation}
	where $h(\cdot)$ can be defined in three ways:	
	\begin{equation} \label{eq:response_dich1}
		h(x) = 
		\begin{cases}
		exp(x)[1 + exp(x)]^{-2} \\
		(2 \pi )^{-1/2} exp(-x^{2}/2) \\
		exp(x - exp(x))
		\end{cases}
	\end{equation}
	which corresponds to the logistic, standard normal $\phi(x)$, and Gumbel (extreme value type I) density  distributions, respectively. In terms of link functions, the distributions corresponds to the well known logit, probit and complementary log-log link functions, respectively. Alternatively, the same parametrization can be achieved using the concept of an underlying latent variable in the form $y^{*} = v + \epsilon^{*}$, where $\epsilon^{*}$ can have a distribution as the ones defined in equation \ref{eq:response_dich1}.
	
	Finally, the distributional part is defined by a Binomial distribution,
	\begin{equation} \label{eq:dist_dich}
		\begin{split}
		f[y=1 | \mathbf{X}, \mathbf{Z}, \pmb{\eta}] &= \binom{n}{k} \mu^{k} (1-\mu)^{n-k} \\
		&= \binom{n}{k} \pi^{k} (1-\pi)^{n-k}
		\end{split}
	\end{equation}

	where $k$ denotes the number of successes in $n$ independent Bernoulli trials.
	
	
			
	\item \textbf{Polytomous:} \\	
	It results from selecting a generalized logistic response function \citep{Bock_1972} for the expected value of the response, which in this case, describe the probability of endorsing one of the $S$ unordered available categories,
	\begin{equation} \label{eq:link_poly}
		\begin{split}
		\mu_{s} &= E[y=y_{s} | \mathbf{X}, \mathbf{Z}, \pmb{\eta}] \\
		&= P[y=y_{s} | \mathbf{X}, \mathbf{Z}, \pmb{\eta}] \\
		& = \pi_{s} \\
		&= h(v_{s})
		\end{split}
	\end{equation}	
	where $v_{s}$ is the linear predictor for category $s$ ($s=1, \dots, S$), and $h(\cdot)$ is defined as:
	\begin{equation} \label{eq:response_poly}
		h(x) = exp(x)\left[\sum_{s=1}^{S} exp(x)\right]^{-1}
	\end{equation}
	It is important to note that, as in the dichotomous case, the same parametrization can be achieved using the concept of underlying continuous responses in the form $y_{s}^{*} = v_{s} + \epsilon_{s}^{*}$, where $y_{s}$ denotes the random utility for the $s$ category, and $\epsilon_{s}^{*}$ have a Gumbel (extreme value type I) distribution, as the one defined in equation \ref{eq:response_dich1}.
	
	Finally, the distributional part is defined by a Multinomial distribution,
	\begin{equation} \label{eq:dist_poly}
		\begin{split}
		f[y=\{y_{1}, \cdots, y_{S}\} | \mathbf{X}, \mathbf{Z}, \pmb{\eta}] &= \frac{n!}{y_{1}! \cdots y_{S}!} \prod_{s=1}^{S} \mu_{s}^{y_{s}} \\
		&= \frac{n!}{y_{1}! \cdots y_{S}!} \prod_{s=1}^{S} \pi_{s}^{y_{s}}
		\end{split}
	\end{equation}
	
	where $y_{s}$ denotes the number of "success cases" in category $s$.

	
	
	\item \textbf{Ordinal and discrete time duration:} \\
	It results from selecting an appropriate response function for the expected value of the manifest variable. For the ordinal case, the expected value corresponds to the probability of endorsing category $s$ against all previous categories, in the following form:
	\begin{equation} \label{eq:link_ord1}
		\begin{split}
			\mu_{s} &= E[y = y_{s} | \mathbf{X}, \mathbf{Z}, \pmb{\eta}] \\
			&= P[y \leq y_{s} | \mathbf{X}, \mathbf{Z}, \pmb{\eta}] - P[y \leq y_{s-1} | \mathbf{X}, \mathbf{Z}, \pmb{\eta}] \\
			&= H(\kappa_{s} - v_{s}) - H(\kappa_{s-1} - v_{s-1})
		\end{split}
	\end{equation}
	where $\kappa_{s}$ denotes the thresholds for category $s$. For discrete time duration, the the expected value corresponds to the probability that the survival time lies in the $s$th interval, as follows:
	\begin{equation} \label{eq:link_ord2}
		\begin{split}
			\mu_{s} &= E[t_{s-1} \leq T \le t_{s} | \mathbf{X}, \mathbf{Z}, \pmb{\eta}] \\
			&= P[T \leq t_{s} | \mathbf{X}, \mathbf{Z}, \pmb{\eta}] - P[T \leq t_{s-1} | \mathbf{X}, \mathbf{Z}, \pmb{\eta}] \\
			&= H(v_{s} + t_{s}) - H(v_{s-1} + t_{s-1})
		\end{split}
	\end{equation}
	where $T$ is the unobserved continuous time, $t_{s}$ is its observed discrete realization. Additionally, $H(\cdot)$ can be defined, for both type of responses, as the cumulative distributions for the logistic, standard normal, and Gumbel (extreme value type I) density  distributions, respectively:
	\begin{equation} \label{eq:response_ord}
	H(x) = 
	\begin{cases}
		exp(x)[1 + exp(x)]^{-1} \\
		\Phi(x) \quad \text{no closed form} \\
		exp(-exp(x))
	\end{cases}
	\end{equation}
	It is important to note that for discrete time duration, the logit link corresponds to a proportional-odds model, while the complementary log-log link to a discrete time hazards model \citep{Rabe_et_al_2001}. Other models for ordinal responses, such as the adjacent category logit model, can be specified as special cases of the generalized logistic response function, observed in equation \ref{eq:link_poly}. 
	
	As with the dichotomous and polytomous case, the same parametrization can be achieved using the concept of underlying latent variables.
	
	Finally, the distributional part is defined by a Multinomial distribution, as the one defined in equation \ref{eq:dist_poly}.
	
	
		
	
	\item \textbf{Counts and continuous time duration:} \\
	It results from selecting a log link function for the expected value of the response,
	\begin{equation} \label{eq:link_count}
		ln(\mu) = ln(E[y | \mathbf{X}, \mathbf{Z}, \pmb{\eta}]) = ln(\lambda) = v
	\end{equation}
	
	and a Poisson distribution for the conditional distribution of the counts,
	\begin{equation} \label{eq:dist_count}
		f[y| \mathbf{X}, \mathbf{Z}, \pmb{\eta}] = exp(-\mu) \mu^{y} (y!)^{-1} = exp(-\lambda) \lambda^{y} (y!)^{-1}
	\end{equation}
	It is important to mention that unlike the models for dichotomous, polytomous and ordinal responses, model for counts cannot be written under the random utility framework, often used in discrete choice models.
	
	\item \textbf{Rankings and pairwise comparisons:} \\
	Following \citet{Rabe_et_al_2003a}, the parametrization for polytomous responses can serve as the building block for the conditional distribution of rankings. Selecting a "exploded logit" function \citep{Chapaaan_et_al_1982} for the expected value of the response, which describes the probability of the full rankings of category $s$,
	
	{\color{red} REVIEW THIS PART
	\begin{equation} \label{eq:link_rank}
		\begin{split}
			\mu_{s} &= P[\mathbf{R}_{s}= \{ r_{s}^{1}, \dots r_{s}^{1}\} | \mathbf{X}, \mathbf{Z}, \pmb{\eta}] \\
			& = \pi_{s} \\
			&= h(v_{s})
		\end{split}
	\end{equation}	
	where $v_{s}$ is the linear predictor for category $s$ ($s=1, \dots, S$), and $h(\cdot)$ is defined as:
	\begin{equation} \label{eq:response_rank}
		h(x) = \prod_{s=1}^{S} exp(x^{s})\left[\sum_{s=1}^{S} exp(x^{s})\right]^{-1}
	\end{equation}

	Again, as in specific previous cases, the same parametrization can be achieved using the concept of underlying latent variables.
	
	Finally, the distributional part is defined by a Multinomial distribution,
	\begin{equation} \label{eq:dist_rank}
		\begin{split}
			f[y=\{y_{1}, \cdots, y_{S}\} | \mathbf{X}, \mathbf{Z}, \pmb{\eta}] &= \frac{n!}{y_{1}! \cdots y_{S}!} \prod_{s=1}^{S} \mu_{s}^{y_{s}} \\
			&= \frac{n!}{y_{1}! \cdots y_{S}!} \prod_{s=1}^{S} \pi_{s}^{y_{s}}
		\end{split}
	\end{equation}
	
	where $y_{s}$ denotes the number of "success cases" in category $s$.
	}


		
	\item \textbf{Mixed responses:} \\
	{\color{red} IMPROVE ON THIS PART
	
	Different links and distributions can be specified for different responses. This allows modeling of left- or right-censored continuous responses by specifying an identity link and normal distribution for uncensored responses and a scaled probit link (with scale equal to the residual standard deviation of the uncensored responses) and binomial distribution otherwise. Mixed responses are common in selection models (e.g., Heckman, 1979) where selection is typically dichotomous but the response of interest is often not. An extension to multilevel selection models is treated in Skrondal, Rabe-Hesketh, and Pickles (2002). Other examples are covariate measurement error problems, for example, logistic regression with measurement errors in a continuous covariate (Rabe-Hesketh, Pickles, and Skrondal, 2003; Rabe-Hesketh, Skrondal, and Pickles, 2003; Skrondal and Rabe-Hesketh, 2003b). In structural equation models with several latent variables, the measurement models for different latent variables may require different links and/or families. Finally, composite links can be useful	for specifying proportional odds models for right-censored responses, for handling missing categorical covariates and many other model types; see Skrondal and Rabe-Hesketh (2004b).
	}


\end{enumerate}

\noindent Additionally, for all of the previous models, heteroscedasticity can be modeled setting random intercepts at level $1$, in the following form:
\begin{equation} \label{eq:het_cont}
	log(\sigma) = \pmb{\alpha}^{T}\mathbf{Z^{(1)}}
\end{equation}

where $\sigma$ is the standard deviation of the errors, and $\mathbf{Z^{(1)}}$ is the design matrix that maps the regression parameters $\pmb{\alpha}$, at level $1$.




\subsection{Structural model for the latent variables}
The structural model for the latent variables has the form:
\begin{equation} \label{eq:structural_model}
	\def\sss{\scriptstyle}
	\setstackgap{L}{12pt}
	\def\stacktype{L}
	\pmb{\eta} = \stackunder{\mathbf{B}}{\sss (M \times M)} \stackunder{\pmb{\eta}}{\sss (M \times 1)} + \stackunder{\pmb{\Gamma}}{\sss (M \times Q)} \stackunder{\mathbf{W}}{\sss (Q \times 1)} + \stackunder{\pmb{\zeta}}{\sss (M \times 1)}
\end{equation}
where $\mathbf{B}$ and $\pmb{\Gamma}$ are parameter matrices that maps the relationship between the latent variables, and the vector of "stacked" covariates $\mathbf{W}$, respectively; $\pmb{\zeta}$ is a vector of errors or disturbances, and $M = \sum_{l} M_{l}$. It is important to notice that while equation \ref{eq:structural_model} resembles to single-level structural equation models, the main difference lies in the fact that the latent variables may vary at different levels.

Additionally, considering that the $\pmb{\eta}$ is permuted appropriately and sorted in increasing order, according to the level of the latent variables, the framework does not allow latent variables to be regressed on latent or observed variables varying at a lower level, as such specifications does not appear to provide {\color{red}valid} modeling options, e.g. {\color{red} put some examples}. Furthermore, the framework does not permit feedback effects among the latent variables, as {\color{red} provide reasons}. Together, the two restrictions imply that the $\mathbf{B}$ is strictly upper diagonal.



\subsection{Distribution of the latent variables}
Finally, to fully specify the framework, we need to define the assumed distributions for the disturbances $\pmb{\zeta}$ or the latent variables $\pmb{\eta}$. If our interest lies in the structural equation model, it is more convenient to specify the distribution of the disturbances; otherwise, we set the distributions for the latent variables. Additionally, as in a multilevel setting, it is assumed that latent variables at different levels are independent, whereas latent variables at the same level may be dependent. In that sense, we consider that all latent variables at level $l$ have a multivariate normal distribution with zero mean and covariance matrix $\Sigma_{l}$, i.e. $\pmb{\eta}^{(l)} \sim MVN(\mathbf{0}, \pmb{\Sigma}_{l})$. However, while the multivariate normal distribution is widely used, any other can be assumed, and even it can be left unspecified by using non-parametric maximum likelihood estimation \citep{Rabe_et_al_2003b}



\subsection{Model identification}
{\color{red} IMPROVE ON THIS PART
	
	The structure of the latent variables is specified by the number of levels L and the number
	of latent variables Ml at each level. A particular level may coincide with a level of clustering in the hierarchical dataset. However, there will often not be a direct correspondence between the levels of the model and the levels of the data hierarchy.
}


\section{Relationship with other modeling schemes}

\citet{Rabe_et_al_2012}

{\color{red} The motivation for multilevel regression models is to handle hierarchical data where elementary units are nested in clusters, such as students in schools, which in
turn may be nested in higher-level clusters (e.g., school districts or states). The latent variables, often called "random effects” in this context, can be interpreted as
the effects of unobserved covariates at different levels that induce dependence among lower-level units. In contrast, the motivation for structural equation models is to
handle variables that cannot be measured directly, and are hence latent, and to model their relationships with each other and with observed or manifest variables.
The latent variables, often called “common factors” inthis context, are measured by manifest variables and induce dependence among them.}


\subsection{Factor Models}

\subsection{Item Response Theory and Generalized Latent Models}

\subsection{Multilevel Models}



\section{Advantages and Disadvantages}