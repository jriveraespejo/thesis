\chapter{Introduction}

\section{Preliminar considerations}

\begin{itemize}
	\item multiple literature about the benefits of re-parametrization, but the procedure is trivial ( only separate random effects), but not on the benefits of a non-centered parametrization
	\item No literature on recovery of IRT parameter of interest with samples below 250. Now is more important as non-centered parametrization might entail further benefits
\end{itemize}


Local independence is one of the key assumptions of Item Response Theory (IRT) models, and it is comprised of two parts: (i) local item independence, and (ii) local individual independence \cite{Baker_2001, Hambleton_et_al_1991a}. In the former case, the assumption entails that the individual's response to an item does not affect the probabilty of endorsing another item, after conditioning on the individual's ability. While in the case of the latter, the assumption considers that an individual's response to an item is independent of another person's response to that same item \cite{Reckase_2009}. 

The literature has shown that IRT models are not robust to the violation of local independence. The transgression of the assumption affects model parameter estimates, inflates measurement reliabilities and test information, and underestimates standard errors (e.g. \citet{Yen_1984, Chen_et_al_1997, Jiao_et_al_2012}, among others). 

However, item response data arising from educational assessments often display several type of dependencies, e.g. testlets, where items are constructed around a common stimulus \cite{Wainer_et_al_2007}; the measurement of multiple latent traits within individuals \cite{Reckase_2009}; cluster effects, where correlation among individuals results from the sampling and measurement mechanism used to gather the data \cite{Raudenbush_et_al_2002}; among others. A good motivating example is the reading comprehension sub-test, from the Peruvian public teaching career national assessment. The test is designed to measure three hierarchically nested sub-dimensions of reading comprehension: literal, inferential and reflective abilities. Furthermore, the items are bundled together in testlets related to a common text or passage. Finally, multiple cluster effects are present, e.g. district and regional clustering.

Recent studies have proposed dual dependence models (DDM) to deal with the teslet and individual clustering dependencies observed in the data \cite{Fujimoto_2020, Fujimoto_2018a, Fujimoto_2018b, Jiao_et_al_2012, Flores_2012, Fox_2010, Reckase_2009, Bradlow_1999}. The majority of these representations have been developed under the bayesian framework, and they are similar in parametrization to multilevel models. On the other hand, an almost independent line of models, known as the Generalized Linear Latent and Mixed Models (GLLAMM) \cite{Rabe_et_al_2004a, Rabe_et_al_2004b, Skrondal_et_al_2004a, Rabe_et_al_2012}, have extended the capabilities for the estimation of multiple latent traits at different hierarchical levels. These developments have been observed mostly under the frequentist framework, and they are similar in parametrization to a hierarchical Structural Equation Model (SEM).

Following the literature of these two sets of models, one can easily notice that both followed a multilevel approach to account for the clustering of persons within the samples (DDM), or the latent structures within the individuals (GLLAMM). Furthermore, DDM use a multidimensional approach to account for the item bundles. However, in some cases their model parametrization differs in a way, that some of them appear to be useful only under their specific contexts. Fortunately, their integration under the bayesian framework is not only trivial, but it can be motivated under either type of models.

The benefits of the integration revolves around two facts: (i) the educational data often presents all of the aforementioned dependencies (as in the motivating example) and more; and (ii) in order to reach appropriate conclusions from the parameter estimates, IRT models need to account for all of these dependencies. The latter is particularly important under the context of policy analysis, as a researcher might be interested in produce inferences at the structural level of the model, i.e. how manifest variables explain the variability in the latent variables, or how the latent variables explain other manifest or latent variables, at different levels.



moreover, no literature have been found on the befits of non-cenetring in psychometrics only in hierrarchical models (betancourt)



{\color{red} (work in progress) \\
	it needs to integrate comments \\
	
The short and long term benefits of effective teaching practices can be observed throughout the literature: improvements in student achievements \citet{Rockoff_2004, Rivkin_et_al_2005, Duflo_et_al_2009, Hanushek_et_al_2012, Muralidharan_et_al_2013, Chetty_et_al_2014a, Araujo_et_al_2016}; development of executive functions \cite{Araujo_et_al_2016}, increased college attendance, higher salaries, and a lower possibility of premature parenthood \cite{Chetty_et_al_2014b}, among others. Similarly, the literature has shown most of the negative impacts resulting from the presence of teacher shortages\footnote{\citet{Bertoni_et_al_2020a} defined it as the context in which the teacher's supply, i.e. the number of available teachers in the system, is less than its demand. The authors further elaborate that one of the causes of these shortages is related to the applicants' lower quality or due to their faulty initial training, implying that the shortage can also be conceived as the lack of good quality teachers. The evidence of such shortage has been more prevalent, but not decisive, with temporary teachers, as they are usually associated with inferior attributes, compared to their contracted counterparts} \cite{Duflo_et_al_2009, Muralidharan_et_al_2013, Duflo_et_al_2015, Ayala_2017, Marotta_2019} or ineffective teaching practices \cite{Hanushek_et_al_2012}.

However, while the evidence have a solid methodological support, \citet{Hanushek_et_al_2006} have indicated that some of the proxy variables, used in the methods, are not consistently related to either teacher effectiveness or quality of instruction, examples of such are: out of field teaching\footnote{\citet{Medeiros_et_al_2018} defines it as teachers that are currently teaching a subject in which they are not specialized or do not have the appropriate certificate.} \cite{Ingersoll_1998, Dee_et_al_2008, Bertoni_et_al_2020a}; teaching hours \cite{Bruns_et_al_2015}; years of experience or educational degree \cite{Rockoff_2004, Rivkin_et_al_2005, Clotfelter_et_al_2006, Clotfelter_et_al_2007, Hanushek_et_al_2012}; among others.

Consequently, given that most of the measured teaching factors are proxies, and that the effects estimated from such variables lack consistency, \citet{Hanushek_et_al_2012} have pointed out that the analysis of teacher effectiveness has largely turned away, from attempts to identify the specific characteristics related to such effectiveness, to focus its attention into measuring the direct effect of teachers in the student outcomes\footnote{The method is known as value-added analysis, and it is based on the perspective that a good teacher is one who consistently gets higher achievement from students after other determinants of such are controlled for. For a more detailed explanation of the method refer to \citet{Scherrer_2011}.}. For that reason, considerable uncertainty is still present in the literature, regarding exactly which aspects of teachers are key for the student's learning and whether those qualities can be measured \cite{Rockoff_2004, Clotfelter_et_al_2006}.

However, because the evidence still largely supports the perception that teachers are the main driver behind the student's learning processes, any educational authority need to have, among their main agenda points, the design of an assessment system that can attract, select, develop, and retain the most effective ones \cite{Elacqua_et_al_2018}, and in order to do so, the definition of an Educational Performance Standard (EPS) is a necessity. With an EPS, rooted in the country's context, the authorities can now set clear expectations about what a "good" teacher should know and know to do \cite{Hincapie_et_al_2020}. 

While the specific requirements for such definition are not easy to identify, the aforementioned authors have hinted that most of them can be largely grouped into two: (i) to have the disciplinary knowledge and pedagogical practices adequate to the classroom characteristics, context and teaching level, and (ii) to display such knowledge and practices in the classroom, using the appropriate material and technological resources available. 

As one can infer from the previous general conditions, and the slew evidence, the disciplinary knowledge is a relevant observable factor, consistently associated with teacher effectiveness and growth in the students' achievement \cite{Santibanez_2006, Clotfelter_et_al_2006, Clotfelter_et_al_2007, Hanushek_et_al_2006, Marshall_2009, Rockoff_et_al_2011, Kane_et_al_2011, Kane_et_al_2012, Ome_2012, Metzler_et_al_2012, Kane_et_al_2013, Araujo_et_al_2016, Bietenbeck_et_al_2018, Estrada_2019}; and in that sense, its measurement should be of interest for any educational authority.

The measurement of knowledge has a myriad of available tools, nevertheless, given that any educational department are bounded by budgetary constraints, valid\footnote{the extend to which a measurement tool is well-founded and accurately corresponds to the real measure \cite{Kelley_1927}} and reliable\footnote{the overall consistency of a measure under consistent conditions.} standardized tests\footnote{Assessment instrument in which the implementation, questions, scoring processes, and interpretations are consistent with a predetermined or typified way. The instrument is usually composed of questions or items that fulfill three conditions: (i) they are polytomous, i.e. they have multiple choices, (ii) the choice categories are nominal, i.e. do not present any specific order, and (iii) there is only one "correct" category or answer \cite{Rivera_2019}} stand out, not only for its cost-effectiveness, but also for its simpler implementation \cite{Hincapie_et_al_2020}; and ojective scoring processes and interpretations.

However, as no instrument is perfect, the subject's knowledge scores resulting  from their use will likely have two main problems. First, they could manifest measurement error \cite{Metzler_et_al_2012}, which would imply that the estimates obtained from them could be an biased reflection of the true effects \cite{Angrist_et_al_1999}. And second, as the score is a composite value, does not allow to test which specific factors leads to a better or worse teacher performance.

These two issues has direct and important policy implications, and devoting effort to appropriately assess and control them, could help the educational authorities to understand, for example: (i) the characteristics of the applicants to the public teaching carrer, (ii) to identify which teachers should be hired, and finally, once they are inside, what the authorities should do to train them \cite{Hanushek_et_al_2012}, (iii) if the scores thresholds used for the selection processes are appropriately set\footnote{Approximately 60\% of the Caribbean and Latin American countries use standardized test scores as part of or as a main teacher selection tool \cite{Hincapie_et_al_2020}.}, to mention a few.

In summary, teachers are one of the main drivers behind the student achievements. However, some of the evidence supporting this claim has been based on proxy variables that are not consistently related to the quality of instruction, or methods that are not concerned with the outline of the teaching factors, responsible for the student's learning. Nevertheless, while the literature still reflects considerable uncertainty on what are the "ingredients for a good teacher", a good amount of evidence has supported the disciplinary and pedagogical knowledge, as relevant components of the teacher effectiveness. Finally, the literature has shown that valid and reliable standardized tests are among the best tools to assess such factors, but also have emphasized that such scores could reflect the teacher's abilities with considerable noise.

}

\section{Objectives}

In the face of the previous evidence, this research will have two main goals. First, to describe the method, estimation procedures, and advantages of the Generalized Linear Latent and Mixed Modeling framework (GLLAMM), developed by \citet{Rabe_et_al_2004a, Rabe_et_al_2004b, Skrondal_et_al_2004a, Rabe_et_al_2012}. And second, tests the policy implications of the method, and its results, in a data composed of large repeated Teacher's standardized educational assessments from Peru. \\

\noindent Specifically, for the first objective of the research, the author expects to appraise: 

\begin{enumerate}
	\item If the framework can fulfill all of our methodological requirements for the analysis of complex educational data; e.g. if it can serve multiple psychometric purposes, like analyzing the quality of items, the calculation of dynamical noise-free "scores" for the disciplinary abilities of the teachers, among others; and
	
	\item What are advantages or disadvantages of the method, with particular emphasis in comparison against structural equations, generalized latent and generalized mixed models.
\end{enumerate}

\noindent For the second objective, the author expects to shed some lights about some key policy decisions related to those large evaluation processes, to mention a few:

\begin{itemize}

	\item What are the general characteristics of the applicants to the public teaching-career?, What is the level of their disciplinary knowledge, and how it evolves?, 
	
	\item Do the initial training or socioeconomic status help to explain the disciplinary knowledge profile of the applicants?,
	
	\item What factors of the disciplinary knowledge are consistently related to a good performance in the classroom?,
	
	\item Are the educational authorities screening the teachers with higher disciplinary knowledge?,
	
	\item Do the instruments guarantee a fair assessment of minority groups with different abilities?,
	
	\item After their selection, what differentiate a contract teacher from a temporary one?, and how these differences could be affecting the students?   
\end{itemize}

\noindent Given the aforementioned goals, the researcher believes the master's thesis contributes to the literature in two aspects: 

\begin{enumerate}
	\item In a the theoretical and methodological sense, as the research is focused on offering an exhaustive description and analysis of the GLLAMM framework; and 
	
	\item In a more practical sense, as it helps to provide evidence on some of key policy decisions that most of Latin America countries are currently facing.
\end{enumerate}

\noindent Finally, it is important to mention, that the computational implementation of the method will be developed in \texttt{Stan} \cite{Stan2020} and \texttt{R} \cite{R2015, RStan2020}.



\section{Organization}

\textbf{Chapter \ref{chap:framework}, The Generalized Linear Latent and Mixed Model}, will describe the model, its components, characteristics, assumptions and properties. Finally, the chapter will assess the benefits of the GLLAMM framework against latent factor, structural equation, item-response theory and multilevel models. \\

\noindent \textbf{Chapter \ref{chap:estimation}, Bayesian estimation}, will describe the bayesian framework, its computational implementation, benefits and main shortcomings, in the context of the model. \\

\noindent \textbf{Chapter \ref{chap:application}, Application}, will describe the instruments, its data collection process, and the "dimensions" under analysis. Finally, the the chapter will showcase: (i) the educational theoretical models considered under the analysis, (ii) the sample design used for obtaining the results, (iii) the priors proposals and their inherited assumptions, and (iv) the results of the analysis. \\

\noindent Finally, \textbf{Chapter \ref{cap:conclusions}, Conclusions}, will discuss the conclusion for the research, under the aforementioned framework, and the policy implications derived from its implementation in a large teacher's assessment process. Finally, it will outline the path of future research that can be derived from the present effort.
