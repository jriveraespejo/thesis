\chapter{Simulation Study} \label{chap:simulation}

\section{Conditions} \label{sec:conditions}

A simulation study was conducted to assess three attributes of the bayesian implementation of the GLLAMM, for dichotomous outcomes:
%
\begin{enumerate}
	%
	\item \textbf{Performance.} The study assessed the performance of the MCMC chains, in terms of achieving stationarity, convergence and good mixing,  under centered (CP) and non-centered parametrization (NCP), respectively.
	%
	\item \textbf{Recovery capacity.} The study evaluated the capacity to recover the parameters of interest, e.g. regression parameters, latent variables and loadings. However, it centered its focus on the recovery of the regression parameters, as they are highly relevant for making appropriate inferences at the individual level.
	%
	\item \textbf{Retrodictive accuracy.} The study appraised the capacity of the implementation to retrodict the data of interest.
	%
\end{enumerate} 

\noindent In this context, a fully crossed design with $3 \times 2 \times 2$ experimental conditions was proposed. 

First, the author used three different samples sizes to generate the data under analysis: $500$, $250$, and $100$. The literature on IRT models present several implementations with samples sizes above $250$, however, few present samples lower than that. The author decided to use a sample size of a $100$ to fill in this gap. Moreover, the decision was also supported by the notion that the change of the posterior sampling geometries could benefit the performance, and recovery capacity of the implementation, under this setting.

Second, as expected, the author used two parametrization of the models: CP and NCP. To the author's knowledge, the IRT literature do not use the change of posterior sampling geometries, as an alternative to improve the performance of the bayesian implementation of said models. This study is set to fill in part of this gap.

Third, the author evaluated the performance, recovery capacity and retrodictive accuracy of a first-order and second-order latent variable models (FOLV and SOLV, respectively). The decision was based on the literature of Confirmatory Factor Analysis (CFA), where before fitting a SOLV model (figure \ref{fig:SOLV_model}), the researcher need to asses if the correlation structure at the first level of the FOLV model (figure \ref{fig:FOLV_model}) justifies the decision.

Therefore, ten ($10$) data sets were generated for each study condition, following the algorithm in section \ref{sect:algorithm}. Each data set resembled responses to $25$ binary scored items, conforming to the SOLV model defined in figure \ref{fig:SOLV_model}. The model was motivated by the hypothesized structure for the reading comprehension sub-test, from the Peruvian public teaching career national assessment (see chapter \ref{chap:application}). The latent structure, regression parameters, and loadings remained unchanged throughout the simulation replicas, to reduce experimental error \cite{Kieftenbeld_et_al_2012}. 

%%%%%%%%%%%%%%%%%%%%%%%%%%%%%%%%%%%%%%%%%%%%%%%%%%%%%%%%%%%%%%%%%%%%%%%

\section{Algorithm} \label{sect:algorithm}

Each data replication was simulated following a six-step procedure. First, the author randomly simulated a collection of pseudo-covariates $\mathbf{W}_{\theta} = [ W_{1}, W_{2}, W_{3}, W_{4} ]$, motivated by a similar information set present in the reading comprehension sub-test. The generated covariates were: (i) a binary ``gender" variable ($W_{1}$), describing males and females, (ii) an integer ``age" variable ($W_{2}$) with range $[30, 65]$, the latter corresponding to the age of retirement, (iii) a three-level categorical ``education" variable ($W_{3}$), indicating the type of education the individual received: institute only, university only, or both; and finally (iv) a four-level categorical ``experience" variable ($W_{4}$), denoting the individual's years of work experience, where the higher the category, the higher the years of experience. Associated with these, the author defined their regression parameters $\mathbf{\Gamma}_{\theta} = [\Gamma_{1}, \Gamma_{2}, \Gamma_{3}, \Gamma_{4}]$ where: (i) $\Gamma_{1} = [\gamma_{m}, \gamma_{f}] = [0, 1]$, for males and females, respectively; (ii) $\Gamma_{2} = -0.01$, indicating that the individuals loose ability with age, in a linear manner; (iii) $\Gamma_{3} = [\gamma_{io}, \gamma_{uo}, \gamma_{b}] = [-0.5, 0.5, 0]$, assuming individuals with university degree have better ability levels, followed by individuals with both educations, and last individuals with institute degrees; (iv) $\Gamma_{4} = [\gamma_{0y}, \gamma_{5y}, \gamma_{10y}, \gamma_{11+y}] = [-0.5, 0, 0.35, 0.5]$, implying experience has decreasing returns on abilities; and finally (v) $\Gamma_{0} = 0$, indicating the absence of an intercept;  

Second, the study simulated the second- and first-order latent variables, corresponding to the reading comprehension ability and its three sub-dimensions: literal, inferential and reflective. Reading comprehension ($\theta^{(3)}_{j}$) was generated from a normal distribution $N( \mu^{(3)}_{j}, \sigma^{(3)}_{\theta} )$, with $\mu^{(3)}_{j} = \pmb{\Gamma}_{\theta} \; \mathbf{W}_{\theta}$, i.e. the linear combination of the simulated covariates and its corresponding regression parameters, and $\sigma^{(3)}_{\theta}=0.5$. On the other hand, the three sub-dimensions were generated from a multivariate normal distribution $MVN( \mu^{(2)}_{j.} , \Sigma^{(2)})$, with $\mu^{(2)}_{j.} = [\mu^{(2)}_{j1}, \mu^{(2)}_{j2}, \mu^{(2)}_{j3}] = [\lambda^{(2)}_{1} \theta^{(3)}_{j}, \; \lambda^{(2)}_{2} \theta^{(3)}_{j}, \; \lambda^{(2)}_{3} \theta^{(3)}_{j} ]$, and $\Sigma^{(2)} = S^{(2)} R^{(2)} S^{(2)}$. In order to ensure the FOLV model in figure \ref{fig:FOLV_model} lead us to the SOLV in figure \ref{fig:SOLV_model}, the author assumed loadings $\pmb{\lambda}^{(2)} = [\lambda^{(2)}_{1}, \lambda^{(2)}_{2}, \lambda^{(2)}_{3}] = [0.95, 0.95, 0.95]$. Lastly, $S^{(2)} = \sigma^{(2)}_{\theta} I$, i.e. a diagonal standard deviation matrix with $\sigma^{(2)}_{\theta} = 0.5$; whereas $R^{(2)} = I$, i.e. an identity correlation matrix, implying the sub-dimensions are independent after accounting reading comprehension.

Third, the author defined five ($5$) common stimulus or texts for the items, where the mean difficulty for the texts $\pmb{\eta}^{(3)} = [\eta^{(3)}_{1}, \eta^{(3)}_{2}, \eta^{(3)}_{3}, \eta^{(3)}_{4}, \eta^{(3)}_{5}] = [-1.50, -0.75, 0, 0.75, 1.50]$; whereas the deviation from said mean difficulties were $\sigma^{(3)}_{\eta} = 0.5$ for all texts. 

Fourth, $25$ items were randomly generated from independent normal distributions $N( \mu^{(2)}_{k}, \sigma^{(2)}_{k} ) $, with $\mu^{(2)}_{k} = \pmb{\eta}^{(3)} \pmb{\alpha}^{(2)} \mathbf{A}$ and $\sigma^{(2)}_{k} = \sigma^{(3)}_{k}$; where $\pmb{\alpha}^{(2)} = \mathbf{1}$, indicating the difficulty of the common stimulus directly explained the difficulty of the items, and $\mathbf{A}$ was a block design matrix that maps the items to its corresponding passage, defined in the previous step. Lastly, the dimensions the items were designed to measure were also generated at random.

Fifth, the author calculated the linear predictor $v_{jkd}$ and probability of endorsing an item $\pi_{jkd}$, for each individual $j$ on item $k$ (belonging to dimension $d$), according to equations (\ref{eq:linear_predictor2}), (\ref{eq:systematic}), and (\ref{eq:response_dich1}), respectively; where the probability was calculated using the logistic inverse-link function.
	
Sixth and last, the outcome $y_{jkd}$ was simulated from a Bernoulli distribution as in equation (\ref{eq:distributional}), with a probability of success calculated as in the previous step.

The code associated with the full simulation process can be found in Appendix \ref{appC2_1:sim}.

%%%%%%%%%%%%%%%%%%%%%%%%%%%%%%%%%%%%%%%%%%%%%%%%%%%%%%%%%%%%%%%%%%%%%%%
%%%%%%%%%%%%%%%%%%%%%%%%%%%%%%%%%%%%%%%%%%%%%%%%%%%%%%%%%%%%%%%%%%%%%%%

\section{Parameter estimation}

\subsection{Models and identification}

%%%%%%%%%%%%%%%%%%%%%%%%%%%%%%%%%%%%%%%%%%%%%%%%%%%%%%%%%%%%%%%%%%%%%%%

\subsection{Computational implementation}

As stated in section \ref{sub_sect:software}, the author will implement the models in \texttt{Stan} \cite{Stan2020} and \texttt{R} \cite{R2015, RStan2020}. Furthermore, according to section \ref{sub_sect:burn}, the current research will use a total of $2,000$ iterations, where $1,000$ of them will be spend on warm-up. Finally, the starting values for the parameters will be sampled from the priors defined in the model, as declared in section \ref{sub_sect:starts}.

%%%%%%%%%%%%%%%%%%%%%%%%%%%%%%%%%%%%%%%%%%%%%%%%%%%%%%%%%%%%%%%%%%%%%%%

\subsection{Prior elicitation}





%%%%%%%%%%%%%%%%%%%%%%%%%%%%%%%%%%%%%%%%%%%%%%%%%%%%%%%%%%%%%%%%%%%%%%%
%%%%%%%%%%%%%%%%%%%%%%%%%%%%%%%%%%%%%%%%%%%%%%%%%%%%%%%%%%%%%%%%%%%%%%%

\section{Evaluation criteria}

As stated in section \ref{sec:conditions}, the study was set to evaluate performance, recovery capacity, and retrodictive accuracy of the bayesian implementation.

First, to assess the performance of the MCMC chains, in terms of achieving stationarity, convergence and good mixing, the author followed the usual visual approach. The approach involved the visual evaluation of: (i) trace plots, for stationarity and convergence, and (ii) trank and autocorrelation plots (ACF), for good mixing. Moreover, the assessment of convergence and good mixing were supported by the \texttt{Rhat} and \texttt{n\_eff} statistics developed in \citet{Gelman_et_al_2014} pp. $284-287$.

Second, to evaluate the recovery capacity for all parameters $\pmb{\Omega} = \{ \pmb{\beta}, \pmb{\Lambda}, \pmb{\Theta}, \pmb{\Psi}, \pmb{\Gamma} \}$, the author used the root mean squared error (RMSE), i.e. the extent of the deviation the posterior means exhibited from the true generating values, in all replicate simulations. Notice, in the case of the FOLV model in figure \ref{fig:SOLV_model} no loading recovery can be assessed. However, using Wright's tracing rules \cite{Beaujean_2014}, we evaluated the implied correlation structure between the sub-dimensions, resulting from having a second-order latent variable. The RMSE was defined as follows:
%
\begin{align}
	%
	RMSE \left( \eta^{(m)}_{k} \right) &=\sqrt{\frac{1}{R} \sum_{r=1}^{R} \frac{1}{H} \sum_{h=1}^{H} \left( \hat{\eta}^{(m)}_{krh} - \eta^{(m)}_{k} \right)^2} \\
	%
	RMSE \left( \theta^{(l)}_{jd} \right) &=\sqrt{\frac{1}{R} \sum_{r=1}^{R} \frac{1}{H} \sum_{h=1}^{H} \left( \hat{\theta}^{(l)}_{jdrh} - \theta^{(l)}_{jd} \right)^2} \\
	%
	%
	RMSE \left( \Gamma_{w} \right) &=\sqrt{\frac{1}{R} \sum_{r=1}^{R} \frac{1}{H} \sum_{h=1}^{H} \left( \hat{\Gamma}_{wrh} - \Gamma_{w} \right)^2} \\
	%
	%
	RMSE \left( \lambda^{(l)}_{d} \right) &=\sqrt{\frac{1}{R} \sum_{r=1}^{R} \frac{1}{H} \sum_{h=1}^{H} \left( \hat{\lambda}^{(l)}_{drh} - \lambda^{(l)}_{d} \right)^2}
	%
\end{align}

Notice that since each bayesian implementation used three chains per replica, the statistics were also summarized over them. Therefore, the ``hat" parameters with index $rh$ described the posterior mean in chain $h=1, \dots, H$, and replica $r=1, \dots, R$, where $H=3$ and $R=10$.

Finally, since IRT models are known to be invariant to the shift of the linear predictor \cite{Baker_et_al_1992, Bock_1972}, i.e. the addition/subtraction of a constant to the abilities/difficulties results in the same probability value, it could happen that we observe substantial differences in the recovery of the parameters, that not necessarily implies a classification error  \cite{Wollack_2002}. 
%
\begin{table}[!h]
	\centering
	\begin{tabular}{ c|c|c|cc|c| } 
		& \multicolumn{2}{c}{Predicted outcome} \\
		\hline
		\hline
		True outcome 	& Positive ($1$) 	& Negative ($0$) 	& Total \\
		\hline
		\hline
		Positive 		& TP 				& FP 				& P \\ 
		Negative 		& FN 				& TN 				& N \\ 
	\end{tabular}
	\caption{Confusion matrix of predicted values}
	\label{tab:confusion_matrix}
\end{table}

Therefore, to avoid this inconvenience, the study will also assess the mean retrodictive accuracy ($\overline{ACC}$), true positive rate ($\overline{TPR}$, sensitivity), and true negative rate ($\overline{TNR}$, specificity) of the model, based on all simulated replicas. Consider the confusion matrix in table \ref{tab:confusion_matrix}, the statistics we calculated as follows:
%
\begin{align}
	%
	\overline{ACC} &= \frac{1}{R} \sum_{r=1}^{R} \frac{1}{H} \sum_{h=1}^{H} \frac{TP_{rh} + TN_{rh}}{ P + N} \\
	%
	\overline{TPR} &= \frac{1}{R} \sum_{r=1}^{R} \frac{1}{H} \sum_{h=1}^{H} \frac{TP_{rh}}{P} \\
	%
	\overline{TNR} &= \frac{1}{R} \sum_{r=1}^{R} \frac{1}{H} \sum_{h=1}^{H} \frac{TN_{h}}{N}
	%
\end{align}

\noindent where the statistic for each replica and chain was calculated using the posterior means of the parameters.

%%%%%%%%%%%%%%%%%%%%%%%%%%%%%%%%%%%%%%%%%%%%%%%%%%%%%%%%%%%%%%%%%%%%%%%
%%%%%%%%%%%%%%%%%%%%%%%%%%%%%%%%%%%%%%%%%%%%%%%%%%%%%%%%%%%%%%%%%%%%%%%

\section{Results}
%
\begin{figure}[h]
	\centering
	\includegraphics[width=0.7\linewidth]{4_FOLV_dag}
	%
	\caption[Directed Acyclig Graph (DAG). First Order Latent Variables model (FOLV).]%
	{Directed Acyclig Graph (DAG). First Order Latent Variables model (FOLV). Circles represent latent variables. Squares represent parameters or parameters for priors. Large Squares represent nesting in specific units.}
	\label{fig:FOLV_model}
\end{figure}
%
\begin{figure}[h]
	\centering
	\includegraphics[width=0.7\linewidth]{4_SOLV_dag}
	%
	\caption[Directed Acyclic Graph (DAG). Second Order Latent Variables model (SOLV).]%
	{Directed Acyclig Graph (DAG). Second Order Latent Variables model (SOLV). Circles represent latent variables. Squares represent parameters or parameters for priors. Large Squares represent nesting in specific units.}
	\label{fig:SOLV_model}
\end{figure}

\subsection{FOLV CP}

\subsection{FOLV NCP}

\subsection{SOLV CP}

\subsection{SOLV NCP}

